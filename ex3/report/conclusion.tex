\section{Conclusion}


To compare the implementation of this implementation with the original Atari
Pong from 1976, it cost \$79.95 in 1976\cite{pongreview}, or \$322.84 in
current dollars.\footnotemark A single chip of the EFM32GG can be bought at
newark.com for \$17, and other components needed for creating almost the same
game chip with this implementation would not be much higher than this price. It
can then be said that a new system made with the hardware in this report, could
cost only 10\% of the original price. A mass produced version would of course
be much cheaper.

\footnotetext{
  \url{http://www.westegg.com/inflation/}
}

The energy usage of the game is quite low, and with a standard Coin Cell
Battery\footnotemark, it can run for 21 hours straight. And if you are not bored
of Pong by then, you should afford to buy a new battery.

\footnotetext{
  \url{https://www.sparkfun.com/products/338}
}
