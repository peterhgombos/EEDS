\section{Description and Methodology}

\subsection{Linux build process}

\subsection{The driver}

\subsection{The game}

The game implemented is Pong. Focus during development was kept on keeping code
clean and maintainable, using correct game programming patterns whenever possible,
while still keeping focus on energy efficiency.

As pong is a 'realtime' game, the amount of time one can spend asleep is reduced
compared to a game like chess, where the game can be entirely gamepad interrupt driven.
To conserve power, the game enters sleep mode as often as possible, and
heavy calculations such as collision detection are avoided whenever possible.

Additional care was taken to redraw as fex pixels as possible during screen updates.

\subsubsection{Structure of the game}

Following established patterns of game programming,
updates of the games internal state are fully decoupled from the rendering thereof.

At the start of each iteration of the game loop,
lag (how much time we haven't accounted for) is calculated, and caught up with.
This allows the game to simulate at a fixed rate even on slower hardware,
reducing only visual performance.

If the game loop finishes ahead of time (using less than MS\_PER\_UPDATE ms),
sleep mode is entered for the remainder (time\_to\_next\_update).

With an internal update frequency target of 24 fps,
ENTER EXAMPLE WITH MICROCONTROLLER DATA HERE.
This allows the program to spend most of its time sleeping.

\begin{lstlisting}[caption={Main game loop}, label={lst:main-game-loop}]
/* Some code ommited for brevity */
void game_loop (void)
{
    lag = 0;
    previous_time = get_time ();

    while (running)
    {
        current_time = get_time ();
        elapsed = current_time - previous_time;
        lag += elapsed;

        while (lag >= MS_PER_UPDATE)
        {
            update ();
            lag -= MS_PER_UPDATE;
        }

        draw_game ();

        time_to_next_update = MS_PER_UPDATE - lag;
        sleep (time_to_next_update);
    }
}
\end{lstlisting}


\subsubsection{Input handling}

Whenever the gamepad changes state, i.e. a button is released or pressed,
a signal is received which in turn invokes an input interrupt handler.

The handler polls the gamepad driver, receiving a byte representing currently
pressed buttons. Input is then parsed to fit the games internal representation
of active keys. This abstraction layer between input and internal state is used
to allow for multiple methods of input, practical during development when a
computer keyboard is used.

Having an allways up to date version of button state cached frees the program
from having to poll the driver for each iteration of the main game loop.

\subsubsection{Graphics}
To display graphics on the screen, the game uses the Linux framebuffer device.
All the functions for initialising and adding graphics to the framebuffer are contained in graphics.s.

The initializing function opens the framebuffer file in Linux, and then maps the file to a region in memory.
This is done as it is easier to access an array than to search back and forth in the file.
However, when this is done, the framebuffer will not update after each write.
The function graphics_update does this, and should be called by the game at each frame refresh.

To actually make graphics, the draw_pixel function will color a given pixel with a given color.
It can be called on a subsequent sequense of pixels to create a more complex graphic.
For instance, draw_rectangle can with the draw_pixel function create any rectangle on the screen.
As all the graphics in the pong implementation in this project are square, to get that real nice retro look, only a rectangle function was made.
There would, however, not be any problem to create other shapes with this implementation.

The graphics code is generic when it comes to color, but again as the retro feel was wanted, only black and green is defined in the program.
It is in the specific sprite drawing functions that the color is set.

\subsubsection{Energy efficiency optimization}
