\subsection{Music playback}

\subsubsection{Song selection}

\subsubsection{Music generation}

Songs are exported to 8-bit wave files at 22.05 kilohertz,
ran through a converter script which generates its struct-wrapped\ref{lst:song-struct} C
representation, and finally included through header files and loaded into memory at startup.

\lstinputlisting[caption={Song struct}, label={lst:song-struct}, linerange={4-9}]{../code/songs.h}

Each integer in the notes array represents the amplitude at that sample point.
The lower frequency (exactly half of 44.1 kilohertz) was chosen to conserve space, as the integer representation of a wave file can be 5 to 10 times the size of the original.

Stereo support is present, and if the field is set to 1, samples will alternatingly be piped to the left and right channel output of the DAC.

\subsubsection{Playback}

Music is played by feeding the DAC a new sample every nth clock cycle, where n is the ratio between the clock speed of the CPU and the requested sample rate (every $ \frac{14 MHz}{22.05 KHz} = 634 $th cycle in our case).

A timer is set up with the previously calculated interrupt period, with its interrupt handler \ref{lst:feeding-the-dac} tasked with pushing the next sound sample of the currently selected song to the DAC.

The timer is suspended initially, but re-enables when a song is to be played (when the FSM receives an $ EV\_PLAY $ event).
Once it has been fed all available samples, the $ EV\_STOP $ event is pushed onto the state machine, suspending our song timer, and returning the system to deep sleep mode.

\lstinputlisting[caption={Feeding the DAC}, label={lst:feeding-the-dac}, linerange={75-101}]{../code/timer.c}

The DAC itself is prescaled by $ 2^{7} $ (the maximum possible) and set to Sample / Hold mode.
The lowered frequency, combined with the DAC core shutting down when not converting samples increases energy efficiency greatly during playback\cite{referencemanual}.

\subsubsection{Further improvements}

