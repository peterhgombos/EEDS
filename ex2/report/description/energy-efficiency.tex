\subsection{Energy efficiency}

\begin{itemize}
  \item Turning off unused gpio buttons/lines
  \item Enabling button timer in response to gpio interrupt
  \item Enabling song timer in response to play event
  \item Using DAC in sample and hold mode
  \item Selecting energy mode 2 when none of the timers are running
  \item Selecting energy mode 1 when playing song or checking buttons
  \item Selecting energy mode 0 when updating state machine
  \item Sleeping after checking the state machine
\end{itemize}

Things we investigated but was too much work:

\begin{itemize}
  \item Use low energy timer (LETIMER) for checking buttons.
  \item Use DMA to push samples to the DAC
\end{itemize}

Both were too much work/too hard but would have saved power.

State machine updates in main loop. This requires a wakeup to energy mode 0, which consumes more power.
This is a considered choice. In a very simple application such as this, where all events that require an
FSM update are generated by a single timer, the FSM can be updated from this interrupt service routine.
However in a more realistic application, events would be generated by several different sources, and
updating the FSM in all of these different places is not desirable.
